Following the arguments of ``Trivializing maps''~\cite{Luscher:2009eq},
to evaluate,
\begin{equation}
	⟨O⟩ = 1/Z ∫ \dd x O(x) e^{-S(x)},
\end{equation}
we perform a change of variable,
\begin{equation}
	x = F(y)
\end{equation}
with a vector functions $F$, we have
\begin{equation}
	⟨O⟩ = \frac{1}{Z} ∫ \dd y \left|\det[J(y)]\right| O\left(F(y)\right) e^{ -S\left(F(y)\right) },
\end{equation}
where the Jacobian matrix,
\begin{equation}
	J(y) = \frac{∂F(y)}{∂y}.
\end{equation}
$F$ has to satisfy,
\begin{itemize}
	\item Injective (1 to 1), from the new integration domain to the old.
	\item Continuously differentiable (or differentiable and have continuous inverse).
\end{itemize}

Rewrite the integral as,
\begin{equation}
	⟨O⟩ = \frac{1}{Z} ∫ \dd y O\left(F(y)\right) e^{ -S\left(F(y)\right) + \ln\left|\det[J(y)]\right| }.
\end{equation}
F is a trivializing map, when
\begin{equation}
	S(F(y)) - \ln\left|\det[J(y)]\right| = \text{constant}
\end{equation}
and our expectation value simplifies to
\begin{equation}
	⟨O⟩ = \frac{1}{Z} ∫ \dd y O\left(F(y)\right).
\end{equation}

In terms of HMC, we add the conjugate momenta, $π$,
and use the equations of motion derived from the Hamiltonian,
\begin{equation}
\label{eq:H}
	ℋ(y,π) = ½π² + S\left(F(y)\right) - \ln\left|\det[J(y)]\right|,
\end{equation}
as
\begin{align}
	\frac{\dd}{\dd t} π &= -\frac{∂}{∂y} ℋ = - J(y) S'\left(F(y)\right) + \tr\left[ J⁻¹ \frac{\dd}{\dd y}J \right], \\
	\frac{\dd}{\dd t} y &= \phantom{-}\frac{∂}{∂y} ℋ = π.
\end{align}
This is separable and can use the usual explicit, symplectic and symmetric
discrete integrators.

Consider a change of variable for $π$ and $y$,
\begin{align}
	π &= J(y) p = J\left(F⁻¹(x)\right) p, \\
	y &= F⁻¹(x),
\end{align}
with the Jacobian matrix of determinant $1$,
\begin{align}
	\J(p,x) &=
		\begin{bmatrix}
		J\left(F⁻¹(x)\right) & \frac{∂}{∂x} J\left(F⁻¹(x)\right) p \\
		0 & \frac{∂}{∂x} F⁻¹(x)
		\end{bmatrix}, \\
	\det[\J] &= 1.
\end{align}
We get a new Hamiltonian from equation~\eqref{eq:H},
\begin{equation}
	\tilde{ℋ}(x,p) = ½ p^† M p + S(x) - \ln\left|\det[J]\right|
\end{equation}
where the positive definite M,
\begin{equation}
	M(x) = J^†\left(F⁻¹(x)\right) J\left(F⁻¹(x)\right)
\end{equation}
is the kernel of the kinetic term considered
Duane et al~\cite{Duane:1986fy,Duane:1988vr}.
